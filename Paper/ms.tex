\documentclass[useAMS,usenatbib]{mn2e}

\voffset=-0.8in

% Packages:
\usepackage{graphicx}
\usepackage{amsmath}
\usepackage{xspace}
\usepackage{dsfont}

% Bold symbols
\renewcommand{\btheta}{\{\theta_i\}}
\newcommand{\bx}{\{x_i\}}

% \onecolumn
%%%%%%%%%%%%%%%%%%%%%%%%%%%%%%%%%%%%%%%%%%%%%%%%%%%%%%%%%%%%%%%%%%%%%%%%%%%%%%

\title[]
{Fast Bayesian Inference for Exoplanet Discovery in Radial Velocity Data}
    
\author[Brewer]{%
  Brendon~J.~Brewer$^{1}$\thanks{bj.brewer@auckland.ac.nz},
  % Courtney?
  \medskip\\
  $^1$Department of Statistics, The University of Auckland, Private Bag 92019, Auckland 1142, New Zealand}

%%%%%%%%%%%%%%%%%%%%%%%%%%%%%%%%%%%%%%%%%%%%%%%%%%%%%%%%%%%%%%%%%%%%%%%%%%%%%%

\begin{document}
             
\date{To be submitted to MNRAS Letters}
             
\maketitle

\label{firstpage}

%%%%%%%%%%%%%%%%%%%%%%%%%%%%%%%%%%%%%%%%%%%%%%%%%%%%%%%%%%%%%%%%%%%%%%%%%%%%%%

\begin{abstract}
\end{abstract}

\begin{keywords}
methods: data analysis
\end{keywords}

%%%%%%%%%%%%%%%%%%%%%%%%%%%%%%%%%%%%%%%%%%%%%%%%%%%%%%%%%%%%%%%%%%%%%%%%%%%%%%

\section{Introduction}
The number of known extrasolar planets has exploded in the last two
decades \citep{}. These discoveries have been driven by improvements in
all of the different techniques used to detect and characterise exoplanets,
including the radial velocity method \citep[e.g.][]{},
the transit method \citep[e.g.]{},
and gravitational microlensing \citep[e.g.][]{}. This paper focuses on the
radial velocity method.

Many authors have considered the problem of inferring the properties of an
exoplanetary system from observational data. With radial velocity data,
the expected signal due to an exoplanet is periodic, and the goal is to
infer the number of planets in the system, as well as their properties such
as orbital periods and eccentricities. Many
different techniques have been proposed. These techniques fall into two main
classes: i) those based on periodograms \citep[e.g.][]{}, ii) those based on
Bayesian inference, to describe the uncertainties probabilistically.

\vspace{-0.5cm}
\section*{Acknowledgements}
It is a pleasure to thank Courtney Donovan (Auckland) for many helpful
conversations about this subject. BJB is partially
supported by the Marsden Fund (Royal Society of New Zealand).

\begin{thebibliography}{99}
\end{thebibliography}



\end{document}

%%%%%%%%%%%%%%%%%%%%%%%%%%%%%%%%%%%%%%%%%%%%%%%%%%%%%%%%%%%%%%%%%%%%%%%%%%%%%%
